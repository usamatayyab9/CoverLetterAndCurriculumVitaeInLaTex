\documentclass[9pt,a4paper]{article}
\usepackage[
colorlinks = true,
linkcolor = blue,
urlcolor  = hrefcolor,
breaklinks = true]{hyperref}

\usepackage{xcolor}
\usepackage{titling}
\usepackage{ragged2e}
\usepackage{graphicx}
\usepackage{enumitem}
\usepackage{tabularx}
\definecolor{hrefcolor}{RGB}{38, 66, 139}

\newcommand{\MyHorizontalLine}{\noindent\rule{\linewidth}{1.5pt}\\}
\newcommand{\MyHorizontalLineWithoutNewLine}{\noindent\rule{\linewidth}{1.5pt}}
\predate{}
\postdate{}
\preauthor{}
\postauthor{}
\title{
	\textbf{Muhammad Usama Tayyab}\\Lahore, Pakistan
}
\date{}
\author{}

\begin{document}
	\maketitle
	\pagenumbering{gobble}
	\begin{justify}
	\begin{tabular}{l  r}
		Phone: &+92 322 4227287\\
		Email:&\href{mailto:usamatayyab9@gmail.com}{usamatayyab9@gmail.com}\\
		LinkedIn Profile:&\href{https://www.linkedin.com/in/usama-tayyab-00457418b/}
		{https://www.linkedin.com/in/usama-tayyab-00457418b/}\\
		GitHub Profile:&\href{https://github.com/usamatayyab9}
		{https://github.com/usamatayyab9}
	\end{tabular}
\end{justify}
\MyHorizontalLine
	\textbf{EDUCATION}
	\begin{itemize}
		\item[]\raggedleft\textbf{Bachelor of Computer Science}\hfill\raggedright\textbf{2015 - 2019}\\
		University of Central Punjab
	\end{itemize}
	
\MyHorizontalLine
	\textbf{PROFESSIONAL EXPERIENCE}
	\begin{itemize}
		%%%%%%%%%%%%%%%%%%%%%%%%%%%%%%%%%%%%%%%%%%%%%%%%%%%%%%%%%%%%%%%%%%%%%%%%
		%%%%%%%%%%%%%%%%%%%%%%%%%%%%%%%%%%%%%%%%%%%%%%%%%%%%%%%%%%%%%%%%%%%%%%%%
		%%%%%%%%%%%%%%%%%%%%%%%%%%%%%%%%%%%%%%%%%%%%%%%%%%%%%%%%%%%%%%%%%%%%%%%%
		%%%%%%%%%%%%%%%%%%%%%%%%%%%BYONYKS EXPERIENCE%%%%%%%%%%%%%%%%%%%%%%%%%%%
		\item \raggedleft\textbf{Software Engineer (C++ developer)}
		\hfill
		\raggedright\textbf{June 2021 - Present}\\
		\textbf{Byonyks Pvt ltd.}
		\item []	\justifying
		In my current role, I develop and maintain software for a PD(Peritoneal Dialysis) machine, focusing on new feature implementation, sensor integration, bug resolution and working with different communications protocols such as RS232, RS485, I2C and more. I manage a multi-processing software with inter-process communication, adhering to a layered architecture. All the development is done in Qt and C++ on embedded linux.
		%%%%%%%%%%%%%%%%%%%%%%%%%%%%%%%%%%%%%%%%%%%%%%%%%%%%%%%%%%%%%%%%%%%%%%%%
		%%%%%%%%%%%%%%%%%%%%%%%%%%%%%%%%%%%%%%%%%%%%%%%%%%%%%%%%%%%%%%%%%%%%%%%%
		%%%%%%%%%%%%%%%%%%%%%%%%%%%%%%%%%%%%%%%%%%%%%%%%%%%%%%%%%%%%%%%%%%%%%%%%
		%%%%%%%%%%%%%%%%%%%%%%%%%%%EBRYX EXPERIENCE%%%%%%%%%%%%%%%%%%%%%%%%%%%
		\item \raggedleft\textbf{Software Engineer (C++ developer)}
		\hfill
		\raggedright\textbf{April 2020 - May 2021}
		\textbf{Ebryx Ltd.}\\
		\item[]\justifying
		 Contributed to a Network Security product implemented in C, C++, Qt, and Python for macOS, Linux, and Windows platforms. Implemented socket programming and SDP protocol to facilitate communication between controllers, gateways, and clients using TCP and UDP layers. Transferred data packets securely using Single Packet Authentication (SPA).
		%%%%%%%%%%%%%%%%%%%%%%%%%%%%%%%%%%%%%%%%%%%%%%%%%%%%%%%%%%%%%%%%%%%%%%%%
		%%%%%%%%%%%%%%%%%%%%%%%%%%%%%%%%%%%%%%%%%%%%%%%%%%%%%%%%%%%%%%%%%%%%%%%%
		%%%%%%%%%%%%%%%%%%%%%%%%%%%%%%%%%%%%%%%%%%%%%%%%%%%%%%%%%%%%%%%%%%%%%%%%
		%%%%%%%%%%%%%%%%%%%%%%%%%%%ITU EXPERIENCE%%%%%%%%%%%%%%%%%%%%%%%%%%%%%%%
		\item \raggedleft\textbf{Research Assistant at CSALT}
		\hfill
		\raggedright\textbf{Sept 2019 - March 2020}\\
		\textbf{Information Technology University}
		\item[]\justifying
		Project: CIPL(Crime Investigation and Prevention Lab). In this project I was given the task of keyword spotting in audio utterances.
	\end{itemize}
\MyHorizontalLine
	\textbf{TOOLS/SKILLS}
	\begin{itemize}
	\item[]
		\begin{tabularx}{1.25\linewidth}{XXX}
			AWS & Linux & Bash\\
			LaTex & Qt/Qml & Git\\
			C++17/20 \& STL &  Github & BitBucket\\
			JIRA
		\end{tabularx}
	\end{itemize}
	\MyHorizontalLine
	\textbf{ACADEMIC/HOBBY PROJECTS}
	\begin{itemize}
		\item \textbf{AI Based Snake Game:}
		\begin{itemize}
			\item[]An AI -based snake which uses three different algorithms for snake movement i.e. Breadth-first search (BFS), Depth-first search (DFS) and A* (heuristic based). Random food for snake is generated every time. Game is over when the snake bites itself. This application is built using external C++ libraries SFML (Simple Fast Multimedia Library) for graphics.The type of algorithm is given at run time via command line argument.
		\end{itemize}
		
		\item \textbf{Minesweeper in C$\#$}
		\begin{itemize}
			\item[]A windows 7 like minesweeper game developed in C$\#$ using windows form platform as a part of course project.
		\end{itemize}
		\item \textbf{Solved all problems of the book “The Modern C++ Challenge: Become an expert programmer by solving real-world problems”:}
		\begin{itemize}
			\item[] I solved all 100 problems in "The Modern C++ Challenge" book, showcasing my passion for problem-solving. I indulged into diverse topics such as primality testing, Collatz conjecture, variadic template arguments, regex, working with time zones in Qt, data structures, multithreading, design patterns, file operations (XML, JSON, PDF), image manipulation, SQLite, cryptography and networking.
		\end{itemize}
		
			
		\item \textbf{Solved 203 problems on LeetCode:}
			\begin{itemize}
			\item[]As a hobby project I solved 203 problems on LeetCode. Which include 98 problems on easy difficulty, 102 problems of medium difficulty and 3 problems of hard difficulty. All the problems solved were related to data structures and algorithms and were programmed in C++17.
			\end{itemize}
		
		\item \textbf{Search Engine of Wikipedia articles:}
		\begin{itemize}
			\item[]In this project all provided Wikipedia articles are first stored in memory then queries are performed. Two different data structures are used to store data. 1. Hash map (unordered\_map) and tree structure (map) which are provided by C++11.\\
			Project description:
			\item[]\raggedright\href{https://docs.google.com/document/d/1kiQhWkGNz5xGj6HbdIeS0 W72_jL0nDDU6HkyUZfqWmU/edit}{https://docs.google.com/document/d/1kiQhWkGNz5xGj6HbdIeS0 W72\_jL0nDDU6HkyUZfqWmU/edit}
		\end{itemize}
		
		\item \textbf{File Compression/Decompression-Huffman algorithm:}
		\begin{itemize}
			\item[]A set of programs are developed which compresses a text file given by the user and then decompresses it. To achieve this Huffman algorithm was used. During compression, the input file is read twice: First iteration file is traversed to generate a Huffman tree and store it on disk. Second iteration is used to compress a file. During decompression first the tree is loaded from the disk then decompression is done.
		\end{itemize}
		
		\item \textbf{Notepad and command prompt (Concept used “generic trees”):}
		\begin{itemize}
			\item[]A Windows-like command prompt in which a user can create, edit, delete directories and files. For storing file and directory structure a custom made tree structure was implemented. Along with a text editor in which a user can edit the text of a file.
		\end{itemize}
		
		\item \textbf{Disk Sort/External Sort:}
		\begin{itemize}
			\item[]In this project a program is developed which sorts an input file that can’t be fit into
			RAM. Program reads the input file in chunks, sort each chunk and writes the sorted chunk in binary file. Then all binary files are merged to generate a sorted output file.
		\end{itemize}
	
	\end{itemize}
	\MyHorizontalLine
\end{document}